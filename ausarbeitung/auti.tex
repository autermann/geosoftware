\section{Wichtige Begriffe}
	\subsection{Repository}
		Unter einem Repository versteht man ein zentrales, auf einem Server 
		lagerndes Archiv, das über die gesamte Versionsgeschichte jeder Datei, 
		die im Repository abgelegt wurde, verfügt.
		Da man in der Regel für jedes Projekt ein eigenes Repository benutzt, 
		ist ein Subversion-Server in der Lage mehrere Repositories zu verwalten.
		
		Zum Bearbeiten der versionierten Dateien lädt man sich eine lokale 
		Arbeitskopie aus dem Repository und lädt anschließend die veränderten 
		Dateien in das Repository.
		
		\subsubsection{Trunk}
			Unter dem Trunk versteht man den Hauptentwicklungszweig eines 
			Projektes.
		\subsubsection{Branch}
			Unter einem Branch versteht man eine Kopie des 
			Hauptentwicklungszweiges
		\subsubsection{Tag}
	\subsection{Revision}
		Im jedem Repository gibt es die sogenannte Revisionsnummer. Beim Anlegen 
		eines Repositories beträgt sie null und wird bei jeder eingereichten 
		Änderung inkrementiert. Für jede versionierte Datei wird zusätzlich die
		Revisionsnummer der letzten Bearbeitung gespeichert.
	 	Durch Angabe einer Nummer lässt sich die Version einer Datei eindeutig 
	 	bestimmen und so auch der Zustand einer Datei oder des gesamten Projektes 
	 	zu einem bestimmten Zeitpunkt wiederherstellen.
	 	
\section{Konflikte}
	\subsection{Merge}
	\subsection{Locks}

\section{Die Kommandozeile}
