\section{Graphical Frontends}
Es gibt eine große Auswahl von graphischen Benutzeroberflächen um auf Subversion-Repositories zuzugreifen. Sie vereinfachen den Umgang mit SVN enorm, da komplett auf die Kommandozeilen-Eingabe verzichtet wird.\\
Drei verschiedene Arten von GUIs werden hier nun kurz vorgestellt:
\subsection{SmartSVN}
SmartSVN ist ein eigenständiges Programm, das auf den Betriebssystemen Linux, UNIX, Mac OS X und Windows lauffähig ist. Im Internet ist eine gratis Demoversion für 30 Tage auffindbar, wenn man allerdings die Vollversion benutzen möchte, muss man dafür zahlen.\\
Neben dem Standalone-Client integriert sich SmartSVN unter Windows zusätzlich auch noch in den Explorer (Abbildung \ref{fig:smart1}). Das Menü bietet eine übersichtliche Dateiverwaltung (Abbildung \ref{fig:smart2}) und optisch gut aufbereitete Vergleiche der Versionen an (Abbildung \ref{fig:smart3}).
Neben dem HTTP-Protokoll, werden auch noch HTTPS, SVN und SVN+SSH unterstützt\footnote{Quelle: \url{http://www.syntevo.com/smartsvn}}.
\begin{figure}[!htb]
        \centering
        \includegraphics[width=.8\textwidth]{1_smartsvn1.png}
        \caption{SmartSVN: Shell-Integration}
        \label{fig:smart1}
\end{figure}
\begin{figure}[!htb]
        \centering
        \includegraphics[width=.9\textwidth]{2_smartsvn2.png}
        \caption{SmartSVN: Dateiverwaltung}
        \label{fig:smart2}
\end{figure}
\begin{figure}[!htb]
        \centering
        \includegraphics[width=.9\textwidth]{3_smartsvn3.png}
        \caption{SmartSVN: Unterschiedskontrolle}
        \label{fig:smart3}
\end{figure}
\subsection{Tortoise SVN}
Tortoise hingegen ist ein kostenloser Windows-Client der unter der GNU GPL Lizenz steht. Es integriert sich ebenfalls in den Windows-Explorer und ist deshalb außerhalb und unabhängig von einer IDE verwendbar. Im Gegensatz zu SmartSVN ist TortoiseSVN aber kein standalone-Client, sondern bietet nur die reine Shell-Extension, die allerdings wesentlich umfangreicher ist (Abbildung \ref{fig:tortoise1}).
Zur Zeit ist die Software in 34 Sprachen verfügbar und es werden standardmäßig die Protokolle http, https, svn, svn+ssh, file und svn+xxx unterstützt\footnote{Quelle: \url{http://tortoisesvn.tigris.org/}, \url{http://de.wikipedia.org/wiki/TortoiseSVN}}.
\begin{figure}[!htb]
	\centering
	\includegraphics[width=.45\textwidth]{4_turtoise1.png}
	\caption{Tortoise: Shell-Extension}
	\label{fig:tortoise1}
\end{figure}  
\subsection{Subclipse SVN}
Subclipse ist die dritte GUI Variante, die nun hier vorgestellt wird. Anders als die beiden vorhergegangenen Programme integriert sich Subclipse direkt in die Entwicklungsumgebung Eclipse.\\
Das Eclipse-Plugin steht unter der EPL-Lizenz und ist unter den Betriebssystemen Linux, Mac OS X und Windows lauffähig. Subclipse unterstützt die folgenden Protokolle: http, https, svn, svn+ssh und file. Neben dem Vorteil die Dateien direkt vom SVN Repository in die IDE (siehe Abbildung \ref{fig:subclipse1}) zu laden, bietet Subclipse zusätzlich ansprechende Features wie z.B. die Generierung von Versionsgraphen.\footnote{Quelle: \url{http://subclipse.tigris.org/}}
\begin{figure}[!htb]
        \centering
        \includegraphics[width=1\textwidth]{5_subclipse1.png}
        \caption{Subclipse: Eclipse Synchronize View}
        \label{fig:subclipse1}
\end{figure}
