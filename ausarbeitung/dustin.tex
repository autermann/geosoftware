\section{Versionierung}%%% dustin

\subsection{Zweck von Versionierung}

Versionierung - eigentlich Versionsverwaltung - besch�ftigt sich mit der Erfassung von �nderungen an Dokumenten. Hierf�r werden, in dem zu verwaltendem Archiv, f�r jede Datei Zeitstempel und Nutzerdaten erfasst, welche zu jedem Zeitpunkt wiederhergestellt werden k�nnen. Somit wird eine l�ckenlose Nachvollziehbarkeit von Text�nderungen gew�hrleistet. Versionsverwaltung ist somit eine Art des Variantenmanagements, welches haupts�chlich in der Industrie eingesetzt wird. Dieses Konzept hat sich auf die Softwareentwicklung durchgesetzt. 

\subsection{Geschichtliches}
SVN entstand als Ersatz f�r CVS (Content Versioning System). Jedoch hatte CVS immense Einschr�nkungen, da es nach den Orten der �nderungen Versionierte und nicht nach der zeitlichen Abfolge. Das brachte die SVN-Entwickler dazu "`Sch�nheitsreperaturen"' an CVS durchzuf�hren. Dies begann im Februar 2000. Daraufhin verbreitete sich Subversion selbst, nachdem die Entwickler von CVS auf SVN umstellten. Die erste stabile Version erschien im Jahr 2004.

\subsection{Aufbau der Versionierung}
In der Architektur von Subversion wurde das alte Schema P: L-T (Projectarchive: Location-Time) von CVS zu P: T-L umgewandelt. Dies schaffte den Vorteil, dass SVN anschaulicher wurde und somit die Erfahrungsgem��e Realit�t an �nderungen besser nachstellen konnte. Hierbei bezieht sich das Versionsschema nicht mehr auf einzelne Dateien - wie in CVS - sondern auf das ganze Projekt. Somit l�sst sich einfacher eine Konkrete Version beschreiben. \\