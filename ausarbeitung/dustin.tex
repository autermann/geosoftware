\section{Versionierung}%%% dustin

\subsection{Zweck von Versionierung}

Versionierung - eigentlich Versionsverwaltung - beschäftigt sich mit der Erfassung von Änderungen an Dokumenten. Hierfür werden, in dem zu verwaltendem Archiv, für jede Datei Zeitstempel und Nutzerdaten erfasst, welche zu jedem Zeitpunkt wiederhergestellt werden können. Somit wird eine lückenlose Nachvollziehbarkeit von Textänderungen gewährleistet. Versionsverwaltung ist somit eine Art des Variantenmanagements, welches hauptsächlich in der Industrie eingesetzt wird. Dieses Konzept hat sich auf die Softwareentwicklung durchgesetzt. 

\subsection{Geschichtliches}
SVN entstand als Ersatz für CVS (Content Versioning System). Jedoch hatte CVS immense Einschränkungen, da es nach den Orten der Änderungen Versionierte und nicht nach der zeitlichen Abfolge. Das brachte die SVN-Entwickler dazu "`Schönheitsreperaturen"' an CVS durchzuführen. Dies begann im Februar 2000. Daraufhin verbreitete sich Subversion selbst, nachdem die Entwickler von CVS auf SVN umstellten. Die erste stabile Version erschien im Jahr 2004.

\subsection{Aufbau der Versionierung}
In der Architektur von Subversion wurde das alte Schema P: L-T (Projectarchive: Location-Time) von CVS zu P: T-L umgewandelt. Dies schaffte den Vorteil, dass SVN anschaulicher wurde und somit die Erfahrungsgemäße Realität an Änderungen besser nachstellen konnte. Hierbei bezieht sich das Versionsschema nicht mehr auf einzelne Dateien - wie in CVS - sondern auf das ganze Projekt. Somit lässt sich einfacher eine Konkrete Version beschreiben. \\