\documentclass{beamer}
\usepackage[utf8]{inputenc}
\usepackage[T1]{fontenc}
\usepackage{ngerman}
\usepackage{hyperref}
\usepackage{multicol}
\newcommand{\itemiz}[1]{\begin{itemize}#1\end{itemize}}

\title{Subversion}
\author{Dustin Demuth\\Stefan Arndt\\Christian Autermann}
\date{\today}


\begin{document}

\frame{
	\titlepage
}

\frame{
	\frametitle{Wozu Versionierung?}
	\begin{itemize}
		\item Erfassung von Änderungen an Dokumenten oder Dateien
		\item Beherrscht: Protokollierung, Archivierung, Wiederherstellung
	\end{itemize}
}

\frame{
	\frametitle{Geschichte}
	\begin{itemize}
		\item RCS [Revision Control System] (1980)
		\item CVS [Concurrent Versioning System](1989)
		\item SVN [Subversion] (2004)
	\end{itemize}
}

\frame{
	\frametitle{Funktionsweise}
	\begin{itemize}
		\item P: L-T vs P: T-L
		\item Örtlich, Zeitlich $\Rightarrow$ Versionierung jedes Inhaltes
		\item Zeitlich, Örtlich $\Rightarrow$ Versionierung des gesamten Projektes 
	\end{itemize}
}

\frame{
	\frametitle{Vorteile von P:T-L}
	\begin{itemize}
		\item Versionierung des gesamten Projektes statt nur einzelner Orte
		\item Vereinfachung der Versionsbeschreibung
		\item Lokale Speicherung älterer Revisionen.
	\end{itemize}
}

\frame{
	\frametitle{Revision}
	\begin{itemize}
		\item Jede Änderung entspricht neuer Revision
		\item Nur geänderte Dateien erhalten neue Nummer
		\item Revision entspricht der letzen Änderung
		\item \emph{head} -- neuste Revision
		\item \emph{base} -- die Revision einer lokalen Arbeitskopie
	\end{itemize}
}	

\frame{
	\frametitle{Repository -- Struktur}
	\begin{description}
		\item[Trunk] Hauptentwicklungszweig
		\item[Branch]
			Nebenentwicklunkgszweige\\
			alternative Konzepte/Ideen		
		\item[Tag]
			Release Candidates\\
			Zwischenstände
	\end{description}
}

\frame{
	\frametitle{Zugriffsmöglichkeiten}
	\begin{itemize}
		\item Lokaler Zugriff
		\item Entfernter Zugriff
		\begin{itemize}
			\item svn://
			\item http://
			\item https://
			\item svn+ssh://	
		\end{itemize}
	\end{itemize}
}

\frame{
	\frametitle{Wichtige Operationen}
	\begin{description}
		\item[Checkout (co)] Abrufen von Dateien
		\item[Commit (ci)] Einreichen von Änderungen
		\item[Add (add)] Versionierung einer neuen Datei
		\item[Update (up)] lokale Dateien auf neuste Revision bringen	
	\end{description}
}

\frame{
	\frametitle{Konflikte}
	\itemiz{
		\item A lädt r21
		\item B lädt r21
		\item B ändert Datei Beispiel.java
		\item A ändert Datei Beispiel.java
		\item A commited r22
		\item B will r22 commiten $\Rightarrow$ Konflikt
	}
}

\frame{
	\frametitle{Merge}
	B "`mergt"' Beispiel.java und commited r23\\
	$\Rightarrow$ nur bei textbasierten Dateien sinnvoll
	\itemiz{
		\item Quellcode
		\item XML
		\item Konfigurationsdateien
		\item \dots
	}
}

\frame{
	\frametitle{Locks}	
	\itemiz{
		\item A fordert Lock für bild.jpg (svn lock bild.jpg)
		\item A bearbeitet bild.jpg
		\item B fordert Lock für bild.jpg $\Rightarrow$ Fehler
		\item A gibt bild.jpg wieder frei (svn unlock bild.jpg)
		\item sinnvoll bei binären Dateien
		\itemiz{
			\item Grafiken
			\item Sound
			\item Video
			\item \dots
		}
	}
}

\frame{
	\frametitle{Kommandozeile}
	\begin{multicols}{3}
		\itemize{
			\item add
			\item blame (praise, annotate, ann)
			\item cat
			\item changelist (cl)
			\item checkout (co)
			\item cleanup
			\item commit (ci)
			\item copy (cp)
			\item delete (del, remove, rm)
			\item diff (di)
			\item export
			\item help (?, h)
			\item import
			\item info
			\item list (ls)
			\item lock
			\item log
			\item merge
			\item mergeinfo
			\item mkdir
			\item move (mv, rename, ren)
			\item propdel (pdel, pd)
			\item propedit (pedit, pe)
			\item propget (pget, pg)
			\item proplist (plist, pl)
			\item propset (pset, ps)
			\item resolve
			\item resolved
			\item revert
			\item status (stat, st)
			\item switch (sw)
			\item unlock
			\item update (up)
		}	
	\end{multicols}
}



\end{document}











