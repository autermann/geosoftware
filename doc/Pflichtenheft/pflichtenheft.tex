\documentclass[a4paper,11pt]{scrartcl}
\usepackage{packages/geosoftware}
\begin{document}

%Titelseite
\title{Geosoftware II \\ \small Pflichtenheft}
\author{Arndt, Autermann, Demuth, Fendrich, Ottenhues, Paluschek}
\date{\today}
\version{1.1.2}
\status{Entwurf}
\authormail{sloth@vespuccis.de}
%Titelseite wird Kreiert
\maketitle
\thispagestyle{empty}

\begin{center}
\bf Vers.: \MyVersion \\
\bf Stadium: \MyStatus\\
\bf Seiten: \thelastpage \\
\bf Kontakt: \email \\
\end{center}
\newpage

\tableofcontents

\newpage
%Ab gehts im Text

\section{Zielbestimmung}
	Dieses Webportal soll der Verwaltung einer Großstadt, Stadt oder Gemeinde zur Verfügung stehen. Dort wird Bürgern die Möglichkeit gegeben auf einfache Art und Weise Mitteilungen an die Stadtverwaltung zu richten. Der Bürger muss so nicht mehr per Telefonat oder E-Mail-Nachrichtenwechsel den richtigen Ansprechpartner für seine Mitteilung  finden. 

	Als Mitteilungen können Beobachtungen in Form von Verschmutzung, Verunreinigung, Vandalismus oder ähnlichen Vorgängen in das Webportal eingetragen werden. Die Beobachtung soll durch Angabe des Stadtteils, der Straße und der Hausnummer oder durch Makierung auf einer Karte spezifiziert werden. Außerdem soll die Beobachtungszeit angegeben werden.

	Mit Hilfe eines Web Feature Service werden die gemeldeten Beobachtungen auf einer Karte repräsentiert, sodass intuitiv Beobachtungen abgefragt werden können.

	Allgemein ist eine Nutzbarkeit auch in jedem anderen Anwendungsbereich möglich, in denen der Verwaltung ortsgebundene Mitteilungen gemacht werden sollen. Eine einfache Anpassung durch Definition von möglichen Meldungen per Konfigurationsdateien oder  Administrations-Interface wird möglich sein.
	\subsection{Musskriterien}
		\begin{itemize}
			\item Den Benutzern muss die Möglichkeit gegeben werden, sich in einem Webportal zu registrieren und dort raumbezogene Daten einzugeben. Diese Daten können sowohl in Adressform als auch interaktiv über eine Karte eingegeben werden. Nachdem die Daten eingetragen worden sind, können Sie zeitgleich auf der Seite des Webportals visualisiert werden.
	            \item Über einen seperaten Administrationsbereich wird dem Besitzer die Möglichkeit gegeben, die eingegebenen Daten und die registrierten Benutzer zu verwalten. 
		\end{itemize}
	\subsection{Sollkriterien}
		\begin{itemize}
			\item In dem Administrationsbereich können die Anwendungsmöglichkeiten, also die Art der eingegebenen 
                    Beobachtungen genauer definiert werden.
			\item Das Webportal bietet eine intuitive Benutzerführung.
			\item Der Server bietet einen WFS zum externen Zugriff auf die Daten.
			\item Den Benutzern wird die Möglichkeit geboten, andere Meldungen als Missbrauch zu kennzeichnen.
		\end{itemize}
	%\subsection{Kannkriterien}
	\subsection{Abgrenzungskriterien}
		\begin{itemize}
			\item Die Eingabemöglichkeiten werden auf die jeweiligen Stadtgrenzen beschränkt.
		\end{itemize}
\section{Produkteinsatz}
	\subsection{Anwendungsbereiche}
		\begin{itemize}
			\item Den Bürgern der Stadt soll die Möglichkeit gegeben werden Beobachtungen in Form von Verschmutzung, Verunreinigung, Vandalismus oder ähnliche Vorgänge mit Hilfe des Webportals zu melden.
			%\item Graffiti
			%\item Vandalismus
			%\item Verkehrssünder
			%\item Blitzer
			%\item
		\end{itemize}
	\subsection{Zielgruppen}
		\begin{itemize}
			\item Bürger -- als Nutzer des Webportals
			\item Verwaltung -- für adminstrative Tätigkeiten
		\end{itemize}
	\subsection{Betriebsbedingungen}
		\subsubsection{physikalische Umgebung des Systems}
			\begin{itemize}
				\item Server(-raum)
				\item Client: Ortsunabhängig
			\end{itemize}
		\subsubsection{Tägliche Betriebszeit}
			Kontinuierlicher Betrieb.
		\subsubsection{Ständige Beobachtung des Systems durch Bediener}
			Nicht notwendig.
		\subsubsection{Unbeaufsichtigte Installation / Unbeaufsichtigter Betrieb}
			\begin{itemize}
				\item Die Installation findet durch bereitgestelltes Fachpersonal statt.
				\item Zur regelmäßigen Wartung wird das Personal bei der Auslieferung durch Fachpersonal geschult.
			\end{itemize}

%\section{Produktübersicht}


%%Hier beginnen die LF
\section{Produktfunktionen}
	\subsection{Benutzerfunktionen}
		\subsubsection{Nutzermanagement}
			\paragraph{LF31110}
				Nutzer können sich unter Angabe von Name und E-Mail-Adresse beim System registrieren.
				Die eingegebenen Daten werden in einer Datenbank gespeichert.
			\paragraph{LF31120}
				Die Registrierung von Nutzern erfolgt über ein Formular via Web-Browser.
			\paragraph{LF31130}
				(Erst) Nach erfolgreicher Registrierung können die Nutzer sich gegenüber dem System identifizieren und Beobachtungen eingeben.
		\subsubsection{Eingabe von Beobachtungen}
			\paragraph{LF31210}
				Identifizierte Nutzer können Beobachtungen, die sie gemacht haben über ein Formular eingeben.
			\paragraph{LF31211}
				Identifiziere Nutzer können bei Mißbrauch Beobachtungen anderer Benutzer melden.
			\paragraph{LF31220}
				Bei der Eingabe von Beobachtungen wird die räumliche Position der Beobachtung mit eingegeben.		
			\paragraph{LF31230}
				Bei der Eingabe der räumlichen Position der Beobachtung kann der Nutzer entweder manuell entsprechende Koordinaten eingeben oder die Position über eine Kartendarstellung auswählen.		
			\paragraph{LF31240}
				Bei der Eingabe von Beobachtungen wird der Zeitpunkt der Beobachtung mit eingegeben.			
			\paragraph{LF31250}
				Bei der Eingabe von Beobachtungen wird das beobachtete Phänomen (z.B. Verschmutzung) mit eingegeben.
			\paragraph{LF31260}
				Bei der Eingabe von Beobachtungen wird der Beobachter mit erfasst.
			\paragraph{LF31270}
				Benutzer erhalten eine Rückmeldung, ob die Eingabe einer Beobachtung erfolgreich war.
			\paragraph{LF31280}
				Die Eingabe von Beobachtungen erfolgt über ein Formular oder interaktive Karte via Web-Browser.
			\paragraph{LF31290}
				Die Eingabe von Beobachtungen soll sowohl numerische Werte (inkl. Maßeinheit) und textuelle Beschreibungen erlauben.
		\subsubsection{Abrufen von Beobachtungen}
			\paragraph{LF31310}
				Benutzer können eine Kartendarstellung abrufen, welche die Positionen der im System erhaltenen Beobachtungen darstellt.
			\paragraph{LF31320}
				Benutzer können über die Kartenansicht einzelne Beobachtungen abrufen und den Inhalt dieser Beobachtung anzeigen lassen.
			\paragraph{LF31330}
				Bei der Anzeige des Inhalts einer Beobachtung sind ihr Wert, Zeitpunkt, Raumbezug, das beobachtete Phänomen und Beobachter anzugeben.
			\paragraph{LF31340}
				Der Benutzer kann innerhalb der Kartenansicht zoomen.
			\paragraph{LF31350}
				Der Benutzer kann den dargestellten Kartenausschnitt verschieben.
			\paragraph{LF31360}
				Die Kartendarstellung ermöglicht die Einbindung topographischer Karten.
			\paragraph{LF31370}
				Der Benutzer kann die Kartendarstellung mit Hilfe eines Web-Browsers aufrufen.
		\subsubsection{Administratorfunktion}
			\paragraph{LF32100}
				Der Administrator kann auf eine Liste aller registrierten Benutzer zugreifen.
			\paragraph{LF32200}
				Der Administrator kann Benutzer sperren, editieren und aus dem System entfernen.
			\paragraph{LF32300}
				Der Administrator kann Beobachtungen aus dem System entfernen und editieren.
			\paragraph{LF32400}
				Der Administrator kann gemeldete Beobachtungen einsehen und gemäß LF32300 / LF32200 bearbeiten.

%laber zeugs
%\section{Produktdaten}
%\section{Produktleistungen bezüglich Zeit und Genauigkeit}
%\section{Qualitätsanforderungen}
\section{Grafische Benutzeroberfläche}
	\subsection{Server}
		Der Server wird über keine grafische Benutzeroberfläche verfügen. Die Administration erfolgt über Konfigurationsdateien und die Client-Applikation.
	\subsection{Client}
		Die Oberfläche des Client ist eine dreigeteilte Website. Am oberen Rand wird sich eine Leiste zur Benutzerverwaltung. Im unangemeldeten Zustand umfasst sie Felder zum Eintragen von Benutzername und Passwort und die die Button \glqq Anmelden\grqq\ und \glqq Registrieren\grqq . Im angemeldeten Zustand enthält sie einen Button zum Abmelden und ein Anzeige des momentanen Status (bspw. der Benutzername). Für Administratoren enthält sie zudem einen Link zum Administratorpanel, für normale Benutzer einen Link zum Ändern der eigenen Einstellungen (Account löschen, E-Mail/Passwort ändern, etc.).\\
		Der untere Abschnitt wird zweigeteilt. Auf der linken, schmaleren Seite befindet sich ein Suchfeld um bereits eingetragene Beobachtungen zu finden. Diese werden darunter textuell dargestellt. Jeder Beschreibung ist ein “Link” beigefügt, der die Hervorhebung des Ereignisses auf der rechts gelegenden Karte ermöglicht. Andersherum wird beim Auswählen einer Beobachtung in der Karte die textuelle Beschreibung angezeigt. Für Administratoren wird das Löschen der Beobachtung sowohl aus der Liste als auch aus der Karte heraus ermöglicht. Für normale Benutzer wird eine Melde-Funktion bereit gestellt. Im linken Abschnitt wird sich außerdem ein Button zum Erstellen einer neuen Beobachtung finden.\\
		Das Admin-Panel ermöglicht die Auflistung der Beobachtungen und registrierten Benutzer und ermöglicht eine Filterungs- und Editierfunktion.\\
		Alle Funktionaltitäten werden mit HTML und JavaScript umgesetzt.
		
\section{Nichtfunktionale Anforderungen}
	Bei den Client-Komponenten ist sicherzustellen, dass sie auf jedem Webbrowser dargestellt werden können, die sich an den XHTML-Standard des W3C halten.
\section{Technische Produktumgebung}
	\subsection{Software für Server und Client}
		Server:
		\begin{itemize}
			\item JRE 1.6
			\item Apache Tomcat 6.x
			\item MySQL 5.1
		\end{itemize}
		Client:
		\begin{itemize}
			\item Grafik- und JavaScript-fähiger Webbrowser
		\end{itemize}
	\subsection{Hardware für Server}
		Den Anforderungen der Software entsprechend.
	\subsection{Orgware, organisatorische Rahmenbedingungen}
		Wird durch die Entwickler gestellt.
	\subsection{Produktschnittstellen}
		Implementierung eines WFS zum Zugriff auf eingetragene Daten.
\section{Spezielle Anforderungen an die Entwicklungsumgebung}
	Der Entwickler stellt eigene Hardware und Software bereit.
\section{Gliederung in Teilprodukte}
	Es wird ein Gesamtpaket bereit gestellt.

%\section{Ergänzungen}
\section{Glossar}
\end{document}
