\documentclass[a4paper,11pt]{article}             % bestimmt das Aussehen eines Dokuments
\usepackage{packages/geosoftware}
\begin{document}

%Titelseite
\title{\%TITLE\% \\ \small --Pflichtenheft--}
\author{Dustin Demuth, Christian Paluschek, Christian Autermann, Christoph Fendrich, Simon Ottenhues, Stefan Arndt}
\date{\today}
\version{0.0.0}
\status{}
\authormail{christian@autermann.org}
%Titelseite wird Kreiert
\maketitle
\thispagestyle{empty}

\begin{center}
\bf Vers.: \MyVersion \\
\bf Stadium: \MyStatus\\
\bf Seiten: \thelastpage \\
\bf Kontakt: \email \\
\end{center}
\newpage

\tableofcontents

\newpage
%Ab gehts im Text

\section{Zielbestimmung}
	\subsection{Musskriterien}
	\subsection{Sollkriterien}
	\subsection{Kannkriterien}
	\subsection{Abgrenzungskriterien}
\section{Produkteinsatz}
	\subsection{Anwendungsbereiche}
	\subsection{Zielgruppen}
	\subsection{Betriebsbedingungen}
		\subsubsection{physikalische Umgebung des Systems}
		\subsubsection{tägliche Betriebszeit}
		\subsubsection{ständige Beobachtung des Systems durch Bediener}
		\subsubsection{Unbeaufsichtigte Installation / Unbeaufsichtigter Betrieb}


%\section{Produktübersicht}


%%Hier beginnen die LF
\section{Produktfunktionen}
	\subsection{Benutzerfunktionen}
		\subsubsection{Nutzermanagement}
			\paragraph{LF31110}
			\paragraph{LF31120}
			\paragraph{LF31130}
			
		\subsubsection{Eingabe von Beobachtungen}
			\paragraph{LF31210}
			\paragraph{LF31220}
			\paragraph{LF31230}
			\paragraph{LF31240}
			\paragraph{LF31250}
			\paragraph{LF31260}
			\paragraph{LF31270}
			\paragraph{LF31280}
			\paragraph{LF31290}
			
		\subsubsection{Abrufen von Beobachtungen}
			\paragraph{LF31310}
			\paragraph{LF31320}
			\paragraph{LF31330}
			\paragraph{LF31340}
			\paragraph{LF31350}
			\paragraph{LF31360}
			\paragraph{LF31370}
		
		\paragraph{Administratorfunktion}
			\paragraph{LF32100}
			\paragraph{LF32200}
			\paragraph{LF32300}



%laber zeugs
\section{Produktdaten}
\section{Produktleistungen bezüglich Zeit und Genauigkeit}
\section{Qualitätsanforderungen}
\section{Grafische Benutzeroberfläche / Benutzungsoberfläche und Zugriffsrechte}
\section{Nichtfunktionale Anforderungen}
\section{Technische Produktumgebung}
	\subsection{Software für Server und Client, falls vorhanden}
	\subsection{Hardware für Server und Client getrennt}
	\subsection{Orgware, organisatorische Rahmenbedingungen}
	\subsection{Produktschnittstellen}
\section{Spezielle Anforderungen an die Entwicklungsumgebung}
	\subsection{Software}
	\subsection{Hardware}
	\subsection{Orgware}
	\subsection{Entwicklungsschnittstellen}
\section{Gliederung in Teilprodukte}
\section{Ergänzungen}
\section{Glossar}



\end{document}