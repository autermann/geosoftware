\documentclass[a4paper,11pt]{article}             % bestimmt das Aussehen eines Dokuments
\usepackage{packages/geosoftware}
\begin{document}

%Titelseite
\title{Geosoftware II \\ \small Pflichtenheft}
\author{D. Demuth, C. Paluschek, C. Autermann, C. Fendrich, S. Ottenhues, S. Arndt}
\date{\today}
\version{0.0.1}
\status{under construction}
\authormail{christian@autermann.org}
%Titelseite wird Kreiert
\maketitle
\thispagestyle{empty}

\begin{center}
\bf Vers.: \MyVersion \\
\bf Stadium: \MyStatus\\
\bf Seiten: \thelastpage \\
\bf Kontakt: \email \\
\end{center}
\newpage

\tableofcontents

\newpage
%Ab gehts im Text

\section{Zielbestimmung}
	\subsection{Musskriterien}
		\begin{itemize}
			\item Webportal
			\item Eingabe von raumbezogenen Daten (mit Karte)
			\item Visualisierung von Daten
			\item Administrationsmöglichkeiten
			\item Registrierung/Anmeldung
		\end{itemize}
	\subsection{Sollkriterien}
		\begin{itemize}
			\item Definieren von Anwendungsmöglichkeiten per Admin-Panel
			\item Intuitive Benutzerführung
			\item WFS zum Abgreifen der Daten
			\item Leichte Wartbarkeit
			\item melden von missbrauch
		\end{itemize}
	%\subsection{Kannkriterien}
	\subsection{Abgrenzungskriterien}
		\begin{itemize}
			\item Begränzung auf eine Stadt
		\end{itemize}
\section{Produkteinsatz}
	\subsection{Anwendungsbereiche}
		\begin{itemize}
			\item Müllbeobachtungen, Verschmutzung, Verunreinigung
			%\item Graffiti
			%\item Vandalismus
			%\item Verkehrssünder
			%\item Blitzer
			%\item 
		\end{itemize}
	\subsection{Zielgruppen}
		\begin{itemize}
			\item Bürger
			\item Verwaltung
		\end{itemize}
	\subsection{Betriebsbedingungen}
		\subsubsection{physikalische Umgebung des Systems}
			\begin{itemize}
				\item Server(-raum)
				\item Client: ortsunabhängig
			\end{itemize}
		\subsubsection{tägliche Betriebszeit}
			24/7
		\subsubsection{ständige Beobachtung des Systems durch Bediener}
			nö
		\subsubsection{Unbeaufsichtigte Installation / Unbeaufsichtigter Betrieb}
			\begin{itemize}
				\item nein/nein
				\item Installation durch Fachpersonal
			\end{itemize}

%\section{Produktübersicht}


%%Hier beginnen die LF
\section{Produktfunktionen}
	\subsection{Benutzerfunktionen}
		\subsubsection{Nutzermanagement}
			\paragraph{LF31110}
				+ in Datenbank gespeichert
			\paragraph{LF31120}
				übernehmen
			\paragraph{LF31130}
				auch übernehmen

		\subsubsection{Eingabe von Beobachtungen}
			\paragraph{LF31210}
				Identifizierte Nutzer können Beobachtungen, die sie gemacht haben über ein Formular eingeben.
			\paragraph{LF31211}
				Identifiziere Nutzer können bei Mißbrauch Beobachtungen anderer Benutzer melden.
			\paragraph{LF31220}
				Bei der Eingabe von Beobachtungen wird die räumliche Position der Beobachtung mit eingegeben.		
			\paragraph{LF31230}
				Bei der Eingabe der räumlichen Position der Beobachtung kann der Nutzer entweder manuell entsprechende Koordinaten eingeben oder die Position über eine Kartendarstellung auswählen.		
			\paragraph{LF31240}
				Bei der Eingabe von Beobachtungen wird der Zeitpunkt der Beobachtung mit eingegeben.			
			\paragraph{LF31250}
				Bei der Eingabe von Beobachtungen wird das beobachtete Phänomen (z.B. Temperatur, Verschmutzung, Schadstoffkonzentration, Verkehrsstörung) mit eingegeben.
			\paragraph{LF31260}
				Bei der Eingabe von Beobachtungen wird der Beobachter mit erfasst.
			\paragraph{LF31270}
				Benutzer erhalten eine Rückmeldung, ob die Eingabe einer Beobachtung erfolgreich war.
			\paragraph{LF31280}
				Die Eingabe von Beobachtungen erfolgt über ein Formular via Web-Browser.
			\paragraph{LF31290}
				Die Eingabe von Beobachtungen soll sowohl numerische Werte (inkl. Maßeinheit) und textuelle Beschreibungen erlauben.
		\subsubsection{Abrufen von Beobachtungen}
			\paragraph{LF31310}
				Benutzer können eine Kartendarstellung abrufen, welche die Positionen der im System erhaltenen Beobachtungen darstellt.
			\paragraph{LF31320}
				Benutzer können über die Kartenansicht einzelne Beobachtungen abrufen und den Inhalt dieser Beobachtung anzeigen lassen.
			\paragraph{LF31330}
				Bei der Anzeige des Inhalts einer Beobachtung sind ihr Wert, Zeitpunkt, Raumbezug, das beobachtete Phänomen und Beobachter anzugeben.
			\paragraph{LF31340}
				Der Benutzer kann innerhalb der Kartenansicht zoomen.
			\paragraph{LF31350}
				Der Benutzer kann den dargestellten Kartenausschnitt verschieben.
			\paragraph{LF31360}
				Die Kartendarstellung ermöglicht die Einbindung topographischer Karten.
			\paragraph{LF31370}
				Der Benutzer kann die Kartendarstellung mit Hilfe eines Web-Browsers aufrufen.
		\subsubsection{Administratorfunktion}
			\paragraph{LF32100}
				Der Administrator kann auf eine Liste aller registrierten Benutzer zugreifen.
			\paragraph{LF32200}
				Der Administrator kann Benutzer sperren, editieren und aus dem System entfernen.
			\paragraph{LF32300}
				Der Administrator kann Beobachtungen aus dem System entfernen und editieren.
			\paragraph{LF32400}
				Der Administrator kann gemeldete Beobachtungen einsehen und gemäß LF32300/LF32200 bearbeiten.

%laber zeugs
%\section{Produktdaten}
%\section{Produktleistungen bezüglich Zeit und Genauigkeit}
%\section{Qualitätsanforderungen}
%\section{Grafische Benutzeroberfläche / Benutzungsoberfläche und Zugriffsrechte}




\section{Nichtfunktionale Anforderungen}
	Bei den Client-Komponenten ist sicherzustellen, dass sie auf jedem Webbrowser dargestellt werden können, die sich an den XHTML-Standard des W3C halten.
\section{Technische Produktumgebung}
	\subsection{Software für Server und Client}
		Server:
		\begin{itemize}
			\item JRE 1.6
			\item Apache Tomcat 6.x 
			\item MySQL 5.1
		\end{itemize}
		Client:
		\begin{itemize}
			\item grafikfähiger, JavaScript-fähiger Webbrowser
		\end{itemize}
	\subsection{Hardware für Server}
		Den anforderungen der Software entsprechend.
	\subsection{Orgware, organisatorische Rahmenbedingungen}
		Wird durch die Entwickler gestellt.
	\subsection{Produktschnittstellen}
		Implementierung eines WFS zum Zugriff auf eingetragene Daten.
\section{Spezielle Anforderungen an die Entwicklungsumgebung}
	Der Entwickler stellt eigene Hardware und Software bereit.
\section{Gliederung in Teilprodukte}
	Es wird ein Gesamtpaket bereit gestellt.

%\section{Ergänzungen}
\section{Glossar}
\end{document}